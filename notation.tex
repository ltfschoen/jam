\noindent

\subsection{TODO}\label{sec:todo}
\begin{description}
  \item[$\bullet$ ${E_T}$] TODO
\end{description}

\subsection{State}\label{sec:state}
\begin{description}
  \item[$\bullet$ $\sigma$] State Identifier
  \item[$\bullet$ $\Upsilon$] State Transition Function (STF)
\end{description}

\subsection{Misc}\label{sec:misc}
\begin{description}
  \item[$\bullet$ $y \prec x$] precedes operator, relation to indicate one term may be defined purely in terms of another
  \item[$\bullet$ $\mathcal{U}$] substitute-if-nothing function
  \item[$\bullet$ $i$, $j$] used for numerical indices
  \item[$\bullet$ $\none$] nothing
\end{description}

\subsection{Functions and Operations}\label{sec:functions}
\begin{description}
  \item[$\bullet$ $\exists$] exists
  \item[$\bullet$ $\Longleftrightarrow$] TODO (section 3.2)
  \item[$\bullet$ $\bigwedge_{i=0}^{x-1}$] big wedge
\end{description}

\subsection{Sets}\label{sec:sets}
\begin{description}
  \item[$\bullet$ $x$, $y$] item of a set or sequence
  \item[$\bullet$ $\mathbf{s}$] set
  \item[$\bullet$ $\powset{\mathbf{s}}$] set power (section 3.3)
  \item[$\bullet$ $|\mathbf{s}|$] set cardinality (section 3.3)
  \item[$\bullet$ $f^{\#}$] function applied to all members of a set to yield a new set (section 3.3) 
  \item[$\bullet$ $\disjoint$] set-disjointness relation (section 3.3)
  \item[$\bullet$ $\error$] indicates unexpected failure of an operation or that a value is invalid or unexpected (section 3.3) 
\end{description}

\subsection{Numbers}\label{sec:numbers}
\begin{description}
  \item[$\bullet$ $\N$] denotes the set of naturals including zero
  \item[$\bullet$ $\N_n$] restricts the set of naturals to values less than $n$.
    \\ - Formally, $\N = \{0, 1, \dots \}$ and $\N_n = \{ x \mid x \in \N, x < n \}$
  \item[$\bullet$ $\N_L$] is equivalent to $\N_{2^{32}}$ and denotes the set of lengths
  of octet sequences that must have limited size to be stored practically
  \item[$\bullet$ $\rem$] modulo operator
  \item[$\bullet$ $5 \div 3 = 1 \remainder 2$] remainder of quotient operation
\end{description}

\subsection{Integers}\label{sec:integers}
\begin{description}
  \item[$\bullet$ $\mathbb{Z}$] denotes the set of integers
  \item[$\bullet$ $\mathbb{Z}_{a \dots b}$] denotes the set of integers within the interval $[a, b)$
    \\ - Formally, $\mathbb{Z}_{a \dots b} = \{ x \mid x \in \mathbb{Z}, a \le x < b \}$ (e.g. $\mathbb{Z}_{2 \dots 5} = \{ 2, 3, 4 \}$).
    \\ - $\mathbb{Z}_{a \dots +b}$ denotes the offset/length form of this set, which is a short form of $\mathbb{Z}_{a \dots a+b}$.
\end{description}

\subsection{Dictionaries}\label{sec:dictionaries}
\begin{description}
  \item[$\bullet$ $\dict{\mathrm{K}}{\mathrm{V}}$] denotes a dictionary mapping from domain $\mathrm{K}$ to range $\mathrm{V}$
  \item[$\bullet$ $\mathbb{D}$] set of all dictionaries
  \item[$\bullet$ $(k \mapsto v)$] key-value pair in dictionary
  \item[$\bullet$ $&\mathbb{D} \subset \big \{ \{ (k \mapsto v) \} \big \}$] defines a dictionary as a member of the set of all dictionaries $\mathbb{D}$ and a set of pairs $p = (k \mapsto v)$
  \item[$\bullet$ $\forall \mathbf{d} \in\ &\mathbb{D} : \forall (k \mapsto v) \in \mathbf{d} : \exists! v' : (k \mapsto v') \in \mathbf{d}$] dictionary's members must associate at most one unique value for any key $k$
  \item[$\bullet$ $\forall \mathbf{d} \in \mathbb{D}&: \mathbf{d}[k] \equiv \begin{cases}
    v & \text{if}\ \exists k : (k \mapsto v) \in \mathbf{d} \\
    \none & \otherwise
  \end{cases}$] define the subscript operator for a dictionary $d$
    \\ - Note, assumes the key exists in the dictionary, otherwise the result is undefined and any block relying on it must be considered invalid
  \item[$\bullet$ $\forall \mathbf{d} \in \mathbb{D}&, \mathbf{s} \subseteq K: \mathbf{d} \setminus \mathbf{s} \equiv \{ (k \mapsto v): (k \mapsto v) \in \mathbf{d}, k \not \in \mathbf{s} \}$] define the subtraction operator for a dictionary $d$
  \item[$\bullet$ $\dict{K}{V} &\subset \mathbb{D}$, \\
    $\dict{K}{V} &\equiv \big \{ \{ (k \mapsto v) \mid k \in K \wedge v \in V \} \big \}$ ] denotes a typed dictionary mapping from domain $\mathrm{K}$ to range $\mathrm{V}$ as a set of pairs $p$ of the form $(k \mapsto v)$
  \item[$\bullet$ $\keys{\mathbf{d} \in \mathbb{D}} &\equiv \{\ k \mid \exists v : (k \mapsto v) \in \mathbf{d}\ \}$, \\
  $\mathcal{V}(\mathbf{d} \in \mathbb{D}) &\equiv \{\ v \mid \exists k : (k \mapsto v) \in \mathbf{d}\ \}$] \\
  denotes the active domain (\ie set of keys) of a dictionary $\mathbf{d} \in \dict{K}{V}$,
  using $\keys{\mathbf{d}} \subseteq K$, and range (\ie set of values) $\mathcal{V}(\mathbf{d}) \subseteq V$,
  where since the co-domain of $\mathcal{V}$ is a set, if different keys with equal values appear in the dictionary,
  the set will only contain one such value.
  \item[$\bullet$ $\forall \mathbf{d} \in \mathbb{D}, \mathbf{e} \in \mathbb{D}: \mathbf{d} \cup \mathbf{e} \equiv (\mathbf{d} \setminus \keys{\mathbf{e}}) \cup \mathbf{e}$] dictionaries combined through the union operator $\cup$,
  which prioritizes the right-side operand in the case of a key-collision.
  \item[$\bullet$ $\keys{\mathbf{d}}$] returns active domain (set of keys) of dictionary
  \item[$\bullet$ $\mathcal{V}(\mathbf{d})$] returns range (set of values) of dictionary
\end{description}

\subsection{Tuples}\label{sec:tuples}
\begin{description}
  \item[About] Tuples are groups of values where each item
may belong to a different set
  \item[$\bullet$ $(a, b)$] tuple notation
  \item[$\bullet$ $(\N, \N)$] set of natural pairs
  \item[$\bullet$ $$T = \ltuple\isa{a}{\N}\ts\isa{b}{\N}\rtuple$] tuple with named components
  \item[$\bullet$ $t_a$, $t_b$] access named components of tuple
  \item[$\bullet$] e.g. denote an item $t \in T$ through subscripting its name, so for some $t = \ltup\is{a}{3}\ts\is{b}{5}\rtup$, $t_a = 3$ and $t_b = 5$
\end{description}

\subsection{Sequences}\label{sec:sequences}
\begin{description}
  \item[$\bullet$ $\seq{T}$] set of sequences with elements from set $T$, and defines a partial mapping $\N \to T$
  \item[$\bullet$ $\seq{T}_n$] set of sequences with exactly $n$ elements from set $T$, and defines a complete mapping $\N_n \to T$
  \item[$\bullet$ $\seq{T}_{:n}$] set of sequences with at most $n$ elements
  \item[$\bullet$ $\seq{T}_{n:}$] set of sequences with at least $n$ elements
  % FIXME - why doesn't this compile?
  % \item[$\bullet$ $\mathbf{s}[i]$] access item at index $i$ in sequence $\mathbf{s}$
  \item[$\bullet$ $\mathbf{s}_i$] access item at index $i$ in sequence $\mathbf{s}$
  \item[$\bullet$ $[0, 1, 2, 3]_{\dots2} = [0, 1]$ and $[0, 1, 2, 3]_{1\dots+2} = [1, 2]$] range in a sequence
  \item[$\bullet$ $|\mathbf{s}|$] length of sequence
  \item[$\bullet$ $\mathbf{s}[i]^\circlearrowleft \equiv \mathbf{s}[\,i \rem |\mathbf{s}|\,]$] modulo subscription
  \item[$\bullet$ $\text{last}(\mathbf{s}) \equiv x$] function that returns final element $x$ of a sequence $\mathbf{s} = \sq{..., x}$
  % TODO - up to here
  \item[$\bullet$ $\frown$] sequence concatenation operator
  \item[$\bullet$ $\wideparen{\mathbf{x}}$] concatenate-all operator for sequences of sequences
  \item[$\bullet$ $x \doubleplus i$] element concatenation
\end{description}

\subsection{Boolean \& Octets}\label{sec:bool-octets}
\begin{description}
  \item[$\bullet$ $\mathbb{B}_s$] set of Boolean strings of length $s$
  \item[$\bullet$ $\Y$] set of octet strings ("blobs") of arbitrary length
  \item[$\bullet$ $\Y_x$] set of octet strings of length $x$
  \item[$\bullet$ $\Y_\$$] subset of $\Y$ which are ASCII-encoded strings
  \item[$\bullet$ $\text{bits}(\Y)$] sequence of bits representing octet sequence
\end{description}

\subsection{Cryptography}\label{sec:cryptography}
\begin{description}
  \item[$\bullet$ $\H$] set of 256-bit values from cryptographic functions (equivalent to $\Y_{32}$)
  \item[$\bullet$ $\H^0$] equals $[0]_{32}$
  \item[$\bullet$ $\mathcal{H}(m)$] Blake2b 256-bit hash function
  \item[$\bullet$ $\mathcal{H}_K(m)$] Keccak 256-bit hash function
  \item[$\bullet$ $\mathcal{H}_x(m)$] first $x$ octets of hash
\end{description}

% Example indentation
% \begin{enumerate}[leftmargin=0.2cm]
%   \item This is the first item with indentation.
%   \modifyenum{leftmargin=0.4cm}
%   \item This is a normal item with a \verb|0.4cm| indentation.
%   \modifyenum{leftmargin=0.8cm}
%   \item This is an indented item with a long text that causes a line break. Now, the enumeration is continued and the new line is indented. 
%   \modifyenum{leftmargin=0.4cm}
%   \item This is a normal item with a \verb|0.4cm| indentation.
%   \modifyenum{resume,leftmargin=*}
%   \item Here is a normal flush-left item.
%   \item Here is a normal flush-left item.
%   \item Here is a normal flush-left item.
% \end{enumerate}
